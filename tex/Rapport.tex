\documentclass[12pt,a4paper]{article}

\usepackage{lmodern}
\usepackage[T1]{fontenc}
\usepackage[utf8]{inputenc}
\usepackage[french]{babel}

\usepackage{amsfonts}
\usepackage{amsmath}
\usepackage{amssymb}
\usepackage{amsthm}
\usepackage{color}
\usepackage{fullpage}
\usepackage{graphicx}
\usepackage[cache=false]{minted}
\usepackage{parskip}

\usepackage[top=2cm,bottom=2cm,left=18mm,right=18mm]{geometry}

\author{Lucas David \& Théo Legars}

\title{Rapport de projet de reconnaissance d'images}
\date{}

\begin{document}
\maketitle

\section{Implémentation et utilisation du classificateur à distance minimum (DMIN)}

On choisi d'implémenter ce classificateur sous forme d'une classe, ceci étant le plus courant et le plus pratique pour encapsuler les comportements et stocker les données nécessaires.

% code du fichier dmin.py

L'utilisation se résumera aux lignes de code suivante:

% lignes de code

On peut donc déterminer le taux de réussite via la fonction membre \lstinline|score(<données à tester>, <labels correspondants>)|.

Dans le cas de nos données de développement, on obtient un score de 68,80\% pour une exécution de 96,45 secondes.
Sur l'ensemble d'entraînement il est intérêssant de voir que le score n'est pas 100\% mais moins,


\end{document}