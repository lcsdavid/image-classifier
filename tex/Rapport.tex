\documentclass[12pt,a4paper]{article}

\usepackage{lmodern}
\usepackage[T1]{fontenc}
\usepackage[utf8]{inputenc}
\usepackage[french]{babel}

\usepackage{amsfonts}
\usepackage{amsmath}
\usepackage{amssymb}
\usepackage{amsthm}
\usepackage{color}
\usepackage{fullpage}
\usepackage{graphicx}
\usepackage[cache=false]{minted}
\usepackage{parskip}

\usepackage[top=2cm,bottom=2cm,left=18mm,right=18mm]{geometry}

\author{Lucas David \& Théo Legars}

\title{Rapport de projet de reconnaissance d'images}
\date{}

\begin{document}
\maketitle

\section{Implémentation du classifieur à distance minimum (DMIN)}

On choisi d'implémenter ce classifieur sous forme d'une classe, ceci étant le plus courant et le plus pratique pour encapsuler les comportements et stocker les données nécessaires.

\inputminted{python}{../dmin.py}[breaklines=true]

\end{document}